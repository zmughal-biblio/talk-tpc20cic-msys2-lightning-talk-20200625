%% vim:tw=66:spell:wrap:ft=tex:
\ifx \printpresenthandout \undefined
	\ifx \printpresentarticle \undefined
		% no handout and no article
		\documentclass{beamer}
	\else
		% print article
		\documentclass[11pt]{article}
		\usepackage{beamerarticle}
	\fi
\else
	% handout
	\documentclass[handout]{beamer}
\fi
% Preamble
%\usepackage[small,sf,bf]{titlesec}
\PassOptionsToPackage{table,x11names,dvipsnames,rgb}{xcolor}

\usepackage[utf8]{inputenc}
\usepackage[british]{babel}

\usepackage{xstring}

\usepackage{tikz}
\usetikzlibrary{arrows,shapes,automata,positioning,trees,shadows,mindmap,decorations,calc}
\usepackage{colortbl}

\usepackage{float}
\usepackage{graphicx}
\usepackage[normalem]{ulem}

%% use as
%%     \vertcenterimage{\includegraphics{*}}
\newcommand{\vertcenterimage}[1]{\raisebox{-.5\height}{#1}}

%% use as
%%     \flipbox{\includegraphics{*}}
\newcommand{\flipbox}[1]{\scalebox{1}[-1]{#1}}

\usepackage{amsmath}
\usepackage{amsfonts}
\usepackage{bm} % bold mathematics (\bm command)

% set-builder notation using \Set{ ... | ... }
\usepackage{braket}

\usepackage{mathtools}
\usepackage{breqn}

%% make everything in the box bold; both the text and mathematics: using
%% \boldmath from `bm` package
\newcommand{\be}{\bfseries\boldmath}

\newcommand\computertext[1]{\texttt{#1}}
\newcommand\computertextfamily{\ttfamily}

\usepackage{diagbox} % \backslashbox in tables

\usepackage{pifont}% http://ctan.org/pkg/pifont
\newcommand{\cmark}{\ding{51}}%
\newcommand{\xmark}{\ding{55}}%

\newcommand{\todo}[1]{%
	\textcolor{red}{TODO: #1}%
}
\newcommand{\todofig}[1]{%
	\textcolor{red}{TODO figure: \nolinkurl{#1}}%
}


\usepackage{hyperref}
\hypersetup{%
	pdfauthor={Zakariyya Mughal},%
	pdfpagemode={UseNone},%
	pdfpagelayout={SinglePage}%
}


\usetheme{Ilmenau}
\usecolortheme{beaver}
\usepackage{ifxetex}

\ifxetex
	% set font to Tahoma
	\usefonttheme{professionalfonts} % using non standard fonts for beamer
	\usefonttheme{serif} % default family is serif
	\usepackage{fontspec}
\else
	\usepackage[T1]{fontenc}
\fi

% Title on title slide
\setbeamerfont{title}{size = \Large}
\setbeamercolor{title}{fg = black, bg = white}

\setbeamertemplate{headline}{}
\beamertemplatenavigationsymbolsempty

%\usefonttheme[onlymath]{serif}
\ifx \printpresentarticle \undefined
	\setbeamertemplate{frametitle}[default][center]
	\setbeamertemplate{footline}{}
	%\setbeamertemplate{footline}[frame number]

	%% Change beamer bullets to circles rather than the ball default
	\setbeamertemplate{itemize items}[circle]
	\setbeamertemplate{enumerate items}[circle]
\fi


\usepackage{textcomp}
\usepackage{fancyvrb}
\usepackage{changepage}
\usepackage{multicol}
\usepackage{wasysym}
\usepackage{listings}

\lstset{%basicstyle=\small\ttfamily,
%numbers=left,
%escapeinside=||
}
\newenvironment{indented}{\begin{adjustwidth}{1.5em}{}}{\end{adjustwidth}}

% http://tex.stackexchange.com/questions/12550/changing-default-width-of-blocks-in-beamer/12551#12551
\newenvironment<>{varblock}[2][.9\textwidth]{%
  \setlength{\textwidth}{#1}
  \begin{actionenv}#3%
    \def\insertblocktitle{#2}%
    \par%
    \usebeamertemplate{block begin}}
  {\par%
    \usebeamertemplate{block end}%
  \end{actionenv}}

\ifx \printpresentnote \undefined
% no notes
\else
\setbeameroption{show only notes}
\fi

%% use as
%%     \af{ number of overlay }{ slide label }
%% e.g.,
%%     \af{2}{intro-slide}
\newcommand{\af}[2]{\againframe<#1|handout:0>[noframenumbering]{#2}}

%% TikZ arrows
%% From <https://tex.stackexchange.com/questions/61507/drawing-arrows-in-beamer>
\tikzset{
    myarrow/.style={
        draw,
        fill=orange,
        single arrow,
        minimum height=3.5ex,
        single arrow head extend=1ex
    }
}
\newcommand{\arrowup}{%
\tikz [baseline=-0.5ex]{\node [myarrow,rotate=90] {};}
}
\newcommand{\arrowdown}{%
\tikz [baseline=-1ex]{\node [myarrow,rotate=-90] {};}
}
\newcommand{\arrowright}{%
\tikz [baseline=-0.5ex]{\node [myarrow,rotate=0] {};}
}
\newcommand{\arrowleft}{%
\tikz [baseline=-0.5ex]{\node [myarrow,rotate=180] {};}
}


%% no "Figure" in \caption{}
%% simple caption
\setbeamertemplate{caption}{\raggedright\insertcaption\par}
%% with color
%\setbeamertemplate{caption}{%
%\begin{beamercolorbox}[wd=.5\paperwidth, sep=.2ex]{block
%body}\insertcaption%
%\end{beamercolorbox}%
%}

\usepackage{listings}
\lstset{language=Shell,%
basicstyle=\footnotesize\singlespacing%
}

\usepackage[toc,acronym]{glossaries}

\usepackage{qrcode}


\begin{document}
% {{{ Meta
% meta needs to be in \begin{document} so that tabular works
\title[MSYS2 and You]{MSYS2 and You: Easier Windows Native Dependencies}
\author[Zaki Mughal]{Zakariyya Mughal}
\date[2020 June 25]{2020 June 25 \\[1ex]
\href{https://perlconference.us/tpc-2020-cloud/}{TPC 2020: Conference in the Cloud} \\[2ex]
\qrcode{https://github.com/zmughal/talk-tpc20cic-msys2-lightning-talk}\\[1ex]
\btVFill
%
{\footnotesize\url{https://github.com/zmughal/talk-tpc20cic-msys2-lightning-talk}}
}
% TPC 2020: A Perl & Raku Conference (June 24-26 2020)

%% transitions are made transparent rather than hidden
\setbeamercovered{transparent}
%  - Title slide.%{{{
\frame{\titlepage}
%}}}

\section{Problem}
\begin{frame}\frametitle{\secname}
	\begin{description}
		\item[P0] You have Perl code that needs to run on
			Windows (e.g., user requirements,
			Windows-only software/hardware support).
		\pause
		\item[P1] It can't be done in pure-Perl.
		\pause
		\item[P2] It requires many tricky-to-install
			native dependencies.
	\end{description}
\end{frame}

\section{Terminology}
\begin{frame}\frametitle{\secname}
\begin{itemize}[<+->]
	\item \href{https://www.cygwin.com/}{Cygwin}: POSIX-like environment for Windows (uses
		\texttt{cygwin1.dll}).
	\item \href{http://mingw.org/}{MinGW}: GCC-based toolchain for Windows (32/64-bit)
	\item MSYS: part of the MinGW project; shell and utilities forked from Cygwin a long
		time ago (e.g., run \texttt{./configure}).
	\item \href{https://www.msys2.org/}{MSYS2}: project to combine MinGW with updates from
		Cygwin and a package manager.
	\item Other approaches: \href{https://docs.microsoft.com/windows/wsl}{Windows Subsystem for Linux}.
\end{itemize}
\end{frame}

\section{Who uses MSYS2}
\begin{frame}\frametitle{\secname}
\begin{itemize}
	\item \href{https://cran.r-project.org/bin/windows/Rtools/}{Rtools}
		(starting with R 4.0.0),
	\item \href{https://github.com/oneclick/rubyinstaller2}{RubyInstaller},
	\item \href{https://pygobject.readthedocs.io/en/latest/getting_started.html#windows-getting-started}{PyGObject},
	\ldots
	\item and several other FOSS projects recommend it
		as their build environment.
\end{itemize}
\end{frame}

\section{Installing All Your Base}
\defverbatim\pacmanInstallPerl{
\small
\begin{verbatim}
pacman -S mingw-w64-x86_64-perl mingw-w64-x86_64-toolchain
\end{verbatim}
}
\begin{frame}\frametitle{\secname}
\begin{itemize}
	\item Go to \url{https://www.msys2.org/} and follow the
		installation instructions (or use the
		\href{https://chocolatey.org/packages/msys2}{Chocolatey}
		to automate).
	\pause\item Open a terminal for one of the subsystems:
		\begin{description}
			\item[MSYS] For building code that
				requires the compatibility layer.
			\item[MINGW32] For 32-bit code.
			\item[MINGW64] For 64-bit code. This is
				most likely what you want.
		\end{description}
	\pause\item Install Perl for MINGW64 and the toolchain (for compiling XS)
		\pacmanInstallPerl
\end{itemize}
\end{frame}

\section{Install your packages\ldots for great justice.}
\defverbatim\pacmanSearch{
\small
\begin{verbatim}
pacman -Ss <query>
\end{verbatim}
}
\defverbatim\pacmanInstall{
\small
\begin{verbatim}
pacman -S mingw-w64-x86_64-gstreamer \
  mingw-w64-x86_64-gst-plugins-base \
  mingw-w64-x86_64-gst-plugins-good \
  mingw-w64-x86_64-gobject-introspection
\end{verbatim}
}
\begin{frame}\frametitle{\secname}
\begin{itemize}[<+->]
	\item Find your package by either browsing
		\url{https://github.com/msys2/MINGW-packages} or
		searching at the prompt:
		\pacmanSearch
	\items Examples: SDL2, FFTW, GStreamer.
	\item Install the packages
		\pacmanInstall
\end{itemize}
\end{frame}

\section{Further information}
\begin{frame}\frametitle{\secname}
	\begin{itemize}
		\item Steps on building \texttt{Gtk3.pm}
			\url{http://project-renard.github.io/doc/development/meeting-log/posts/2016/05/03/windows-build-with-msys2/}.
		\item Script that automates installation of
			\texttt{Gtk3.pm}: \url{https://github.com/MaxPerl/perl-Gtk3-windows-installer}.
	\end{itemize}
\end{frame}


\section{Questions / Contact}
\begin{frame}\frametitle{\secname}
Reach out
\begin{itemize}
	\item on IRC: \texttt{sivoais} on \url{irc://irc.perl.org/#native} (we can also chat
		about Alien and FFI!),
	\item on Twitter: \href{https://twitter.com/zmughal}{@zmughal},
	\item on GitHub: \url{https://github.com/zmughal}.
\end{itemize}
\end{frame}


%%% }}}
%%% {{{ END

% Blank frame
%\begin{frame}\frametitle{\secname}
%\end{frame}

\end{document}
